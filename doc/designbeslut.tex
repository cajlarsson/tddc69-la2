\documentclass[a4paper,8pt]{article}
\usepackage[utf8]{inputenc}
\usepackage[T1]{fontenc}      
\usepackage[swedish]{babel}
\usepackage{inconsolata}
\begin{document}

\title{Designbeslut för labb 2}
\author{Caj Larsson\\Erik Lindholm}\maketitle

Vårt första beslut var att överge det givna kodskelettet. I början lade vi ned
mycket tid på att försöka förstå hur det var tänkt att det skulle kunna byggas
ut till något som uppfyller alla krav. Vi valde att börja om från scratch och då
gick det betydligt fortare. Vi kunde då återanvända i princip all egen kod 
utan mycket bearbetning. Vi upplevde att kodsklettet  var designat för ett 
större projekt med mer avancerade arvsstrukturer och vy-hantering. Då vår labb 
inte var så omfattande bedömde vi att kodsklettet snarare skulle bli en last.

Vi valde att ha en klass som enbart instansierar vår kontrollerare som sedan
instansierar allt som behövs. På det sättet kunde vi få en mer modulär kodbas
vilket möjliggör för senare integartion av kontrolleraren i andra projekt.

Det är tänkt att det att det bara skall finnas en kontrollerare. Kontrolleraren
innehåller en modell för vilken den agerar gränsnitt. Vi har valt att ha detta
mellanlager för att hålla modellen så enkel som möjligt då det är troligt att den
byts ut mot en modell med bättre prestanda. Ingen annan klass än kontrolleraren
gör saker med modellen för att vi ska vara säkra på att ingen ändring sker utan
att ritytorna blir uppdaterade, hade vi inte gjort så fanns det risk att alla
fönster inte skulle visa samma sak. 

För att implementera ett observerare/observervabel-mönster så ärver kontrolleraren
från \texttt{Observable} och ritytan implimenterar \texttt{Observer}

Vi funderade på att göra kontrolleraren \texttt{static} för att slippa skicka
runt den till alla objekt som behöver ha en referens till den, det skulle dock
göra koden betydligt mindre återanvändningsbar och skulle tillåta sämre struktur
på omgivande kod.

Modellklassen är i princip en behållare för \texttt{MyShape} objekt med ett
speciellt gränsnitt. Den har även koll på vilken form som är markerad. Hade 
projektet varit större och koden upplevts mer relevant att återanvända så
hade vi implementerat ett \texttt{Interface} för att underlätta eventuellt
byte av modellklass. Vi upplevde att vi bara skrev en stor mängd gränsnitts- 
metoder

Vår fönsterklass skapar en menypanel med de knappar som rör vår formfabrik, 
andra knapparna kräver tillgång till objekt vi inte tycker att den bör 
känna till. Knappen som skapar nya fönster instansierar helt enkelt ett nytt
fönsterobjekt och skickar med den kontrollerar objektet. Här skulle vi kunna
låta kontrolleraren skapa fönstret och dem i en lista för att ha bättre 
kontroll på vad som händer. Emellertid tyckte vi inte att det var relevant i 
detta lilla projekt.

Menypanelen har två privata knappklasser som i sin konstruktor antingen
tar en färg eller en prototypform. I konstruktorn tar den sitt fönsters 
formfabrik som knapparnas metoder ändrar prototypen på. Dekoreringen av panelen
blir väldigt städad och lätt att bygga ut. Vi valde att inte intigrera fabriken
i panelen och att inte intigrera menypanelen i fönstret för att göra den
frivillig.

Ritytan implimenterar \texttt{Observer} den får en formfabrik och sin 
kontrollerare i sin konstruktor och skapar en egen \texttt{ClickListener} och
en \texttt{MouseMotionListener}. Lyssnarna används för att flytta markera och
signalera till formfabriken respektive kontrolleraren när en form ska läggas 
till eller tas bort och även då om en form ska flyttas eller markeras.
Alla instanser av ritytan uppdateras genom observerarlagret så att de alla
visar en uppdaterad verison av modellen i sina respektive ytor.

Alla konkreta formklasser ärver det mesta av sin funktionalitet från 
\texttt{MyShape}, såsom konstruktor en kloningsmetod och metoder för att 
ändra och läsa instansvariabler. Det enda som en konkret formklass behöver 
implementera är en konstruktor som anropar superkonstruktorn samt end 
\texttt{draw} metod som ritar ut den på angiven \texttt{Graphics}. Vi har inte 
kommit på något alternativ till detta som alls går att jämföra med.

Formfabriken nyttjar ett \texttt{Prototype pattern} liknande designmönster
för att skapa nya former. Formfabriken har olika metoder för att ändra
produktens färg och form. Orsaken till att vi inte valde ett 
\texttt{Factory pattern} var för att vi inte ansåg att det inte gick att
uppfylla kraven eller så hade vi fått göra multipla formfabriker så att vi
har en för varje form.

Vi funderade på att låta en \texttt{Timer} ärva från \texttt{Observable} och
låta den skicka medelanden till alla \texttt{MyShape} att de ska uppdatera sitt
animeringsläge, som de då skulle få implementera i underklassen. Då skulle vår
formfabrik registrera alla formobjekt som lyssnare. Även ritytorna behövs läggas
till för att uppdatering ska ske.

Ofta skulle vi kunna använt arv mer för att snygga upp koden, emellertid 
gjorde vi avägningar för att göra koden modulär och följa Javas riktlinjer om
objektorientering.

Vi valde att skriva all kod på engelska då java är uppbyggt på engelska och det
blir då omöjlig att ha all kod på samma språk. Dock valde vi att skriva all
dokumentation på svenska då kursen går på svenska.

\end{document}