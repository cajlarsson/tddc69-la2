\documentclass[a4paper,8pt]{article}
\usepackage[utf8]{inputenc}
\usepackage[T1]{fontenc}      
\usepackage[swedish]{babel}
\usepackage{inconsolata}
\begin{document}

\title{Designbeslut för labb 2}
\author{Caj Larsson\\Erik Lindholm}
\maketitle

Vårt första beslut var att dumpa det givna kodskelettet. Vi lade i början ned 
mycket tid på att försöka förstå hur det var tänkt att den skulle kunna byggas
ut till något som uppfyller alla krav. När vi började om från scratch gick det
betydligt mycket fortare och vi kunde återanvända i princip all vår egna kod 
utan speciellt mycket omarbetning.

Vi valde att ha en klass som enbart instansierar vår kontrollerare som sedan
instansierar allt som behövs.

Kontrolleraren är tänkt att det skall finnas en av. Den innehåller en modell
som den agerar gränsnitt mot, och ärver in \texttt{Observable}. Detta 
mellanlager har vi för att hålla modellen så enkel som möjligt då det är 
troligt att den byts ut mot en med bättre prestanda. Ingen annan klass än 
kontrolleraren gör saker med modellen för att vi ska vara säkra på att ingen
ändring sker utan att ritytorna blir uppdaterade om detta.

Modellklassen är i princip en behållare för \texttt{MyShape} objekt med ett
speciellt gränsnitt. Den har även koll på vilken form man har markerat.

Vår fönsterklass skapar en menypanel med de knappar som rör vår formfabrik, de
andra knapparna kräver tillgång till objekt vi inte tycker att den bör 
känna till. I sin konstruktor lägger den till sin rityta i kontrollerarens 
notifieringslista.

Menypanelen har två stycken privata knappklasser som i sin konstruktor antingen
tar en färg eller en prototypform. I konstruktorn tar den sitt fönsters 
formfabrik som knapparnas metoder ändrar prototypen påå 


\end{document}

